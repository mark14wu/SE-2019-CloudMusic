%!TEX root = ../main.tex

\chapter{总体概述}

\section{软件概述}
\subsection{项目介绍}

如今音乐已成为人们生活中不可或缺的一部分,不论年轻人还是老年人都养成了享受音乐的习惯。

而市面上,已经有许多成熟的音乐播放系统,为了达到产品生存与盈利,我们需要不同点与亮点。
我们的\proname 项目有两大竞争优势。

首先,我们是云音乐,拥有强大的云端服务器,可以使用户随时随地享受自己拥有的音乐。
其次,我们拥有更加智能的AI算法来构建用户画像,能更加精确地推荐用户潜在喜欢的歌曲,提供给用户更加舒适的体验。增加用户粘度与口碑。

\subsection{产品环境介绍}

我们的项目分为服务端与客户端,服务端分布式部署在云服务器上,
    所有客户端都由该服务端进行数据交互与信息支持;
    而为了满足用户多种多样的收听音乐的环境,
    我们的客户端有较多种类,
    目前,我们需要考虑以下客户端环境:
    \begin{enumerate}
        \item \textbf{网页浏览器环境}
        \item \textbf{Windows环境}
        \item \textbf{macOS环境}
        \item \textbf{iOS移动环境}
        \item \textbf{Android移动环境}
    \end{enumerate}

在服务器端,我们需要可靠的数据库和云服务器来支持音乐服务的运行,在这里我们不对数据库的具体实现以及性能要求做限制,仅给出数据库需要满足的结构特性以及功能需求。具体而言,我们需要以下的数据库存储模块:
    \begin{enumerate}
        \item \textbf{音乐数据库},储存了各种音质版本的音乐数据文件;
        \item \textbf{音乐元数据数据库},储存所有音乐的相关信息,如
            作者、所属专辑、发行日期等;
        \item \textbf{商品数据库},需要储存所有音乐的价格以及优惠信息等;
        \item \textbf{用户数据库},需要储存所有用户的信息,包括用户名、密码、密保信息、
            订阅信息、账户余额等;
\iffalse
    \R{
        \item \textbf{用户-用户关系数据库}
                    ,存储用户与用户之间的联系,
            具体而言,储存了用户之间的社交关系,比如好友关系、关注关系、粉丝关系等。
}
\fi
        \item \textbf{用户-商品库},存储用户对音乐的购买信息,包括用户的购物车信息、
            已购买音乐的支付信息与历史记录等;
    \end{enumerate}

服务器端不仅需要数据库服务来存储软件的数据,并且,还需要相关的接口,
    来提供软件来调用,同样,在这一章节,我们不对接口的详细信息做出定义与限制,
    只是给出接口的分类以及服务的功能性描述。
具体而言,我们需要以下的服务器接口:
\begin{enumerate}
    \item \textbf{音乐下载接口},通过该接口,客户端可以获取需要的音乐文件;
    \item \textbf{音乐上传接口},通过该接口,客户端可以上传本地音乐文件致云端;
\iffalse
        \R{
        介于新的需求,我们需要在下载时,进行不同音质的音乐分配。
        并且,根据用户会员的购买情况,音乐的可下载性、可下载的音质都有一定区别。
        }
\fi
    \item \textbf{在线播放服务接口},该接口提供了在线收听没有下载
        的音乐的服务;
\iffalse
        \R{
        介于新的需求,我们同样需要在流播放的过程中进行音质的配置。
        并且,根据用户会员的购买情况,不同音质的可用性将有一定区别。
        }
\fi
    \item \textbf{用户信息交互接口},该接口提供了用户登陆信息的传递以及验证的功能
        (用户注册功能亦然),我们希望该服务可以长期稳定;
    \item \textbf{商品交易服务接口},该接口提供了从客户端购买音乐的功能的基础,
        当用户发起支付的需求时,对于内部支付方式(账户余额)以及多种支付方式可以正确、稳定、高效的处理;
    \item \textbf{第三方API处理接口},该接口主要用于处理我们提供给第三方的外部
        应用程序接口的信息处理
\end{enumerate}

\section{软件功能}

在本小节中,我们对\proname 产品功能细节进行全局介绍。
在这一小节,我们暂时不对功能的具体实现细节以及性能上的限制做出细节的定义,
    只是在全局上给出功能的一个概览,目的是让实现者以及使用者可以
    看到获知我们项目的整体功能概况,对产品需求可以有一个清晰简明的认识。

我们按照用户可以看到的界面来分类这些功能:
\begin{enumerate}
    \item \textbf{主界面}:
    该界面可以进入一系列音乐集(所谓音乐集,对用户而言,称为“歌单”,)。
        在该界面,可以看到以下默认的音乐集的入口:
        \begin{itemize}
            \item 进入``本地音乐''歌单,展示已下载的音乐;
            \item 进入``我的最爱''歌单,展示用户标注为``喜欢''的音乐;
            \item 进入``最近播放''歌单,展示用户最近播放过的音乐;
        \end{itemize}
        在默认歌单下面,有所有用户创建的歌单的入口列表,
        分类为``我的歌单'',并将所有的用户自创歌单按照用户指定的顺序展示。
    \item \textbf{歌单界面}:
    进入一个歌单之后,将展示该歌单的信息和功能按键,应包括:
        \begin{itemize}
            \item 歌单封面
            \item 歌单名称
            \item 歌单创建者,若为专辑歌单,则显示为专辑作者信息
            \item 功能按键栏,包括
            \begin{itemize}
                \item 播放歌单 
                \item 下载歌单 
                \item 编辑歌单(进入编辑模式) 
                \item 分享歌单 
            \end{itemize}
            \item 歌单内的所有音乐
        \end{itemize}
    \item \textbf{音乐播放界面}:
    从搜索功能、歌单内、通过推荐或通过外部链接可以进入音乐播放界面,
        在该界面中,可以控制音乐的播放,以及查看与该音乐有关的信息,
        具体而言,界面中有以下元素:
        \begin{itemize}
            \item 播放/暂停按钮
            \item 播放模式调整按钮:单曲循环、列表循环、顺序播放、随机播放等
            \item 可拖动的进度条
            \item 播放进度,以当前已播放时间和音乐总时间表示
            \item 音乐的相关信息:
            \begin{itemize}
                \item 音乐名称
                \item 音乐的作者信息
                \item 音乐的风格、标签(Tag)
                \item 音乐的收听次
            \end{itemize}
            \item 音乐互动界面(评分、评论)
            \item 查看当前音乐播放列表
            \item 展示音乐封面(或专辑封面)
        \end{itemize}
    \item \textbf{音乐推荐界面}:
        该界面是\proname 的核心功能之一,该界面主要分为两部分:
        \begin{itemize}
            \item 个性推荐模块:
                该模块中,主要根据该用户自身的历史信息以及主观喜好,
                来推荐音乐。
            \item 热门推荐模块:
                该模块中,主要根据全网用户数据来推荐音乐,
                以及展示排行榜。
        \end{itemize}
     \item \textbf{音乐搜索界面}:
    该界面可以进行音乐搜索功能,界面主要有以下元素:
        \begin{itemize}
            \item 搜索栏;
            \item 搜索按键;
            \item 搜索推荐,按照全网搜索次数多少排序;
            \item 搜索历史;
        \end{itemize}。
    \item \textbf{账户界面}:
        在该界面中,若用户没有登陆,则应该显示登陆与注册的功能,
        若已登录,则可以查看账户的相关信息或进行相关操作,包括:
        \begin{itemize}
            \item 展示用户名
            \item 展示头像
            \item 展示账户余额
            \item 展示支付方式的绑定状态
            \item 进入支付历史界面
            \item 进入查看所有已购买音乐界面
            \item 进入设置界面
            \item 编辑账户信息
            \item 退出登录
        \end{itemize}
    \item \textbf{设置界面}:
        在该界面,我们需要能够对软件行为作出一定的设置,该设置需要针对不同的平台与使用场景
            来展示不同的设置选项,一些选项是通用的,包括:
            \begin{itemize}
                \item 播放的音质选项
                \item 自动下载音乐选项
                \item 存储空间相关选项
                \item 播放界面相关选项
                \item 歌词相关选项
            \end{itemize}
\iffalse
    \R{
    \item \textbf{私人空间界面}:
        在该界面,我们需要展示一个用户的基本信息,如昵称、会员状态、头像,以及,
        最重要的,我们需要将社交相关的信息展示在个人空间界面中。
        其中,主要元素分为两大块:
        \begin{enumerate}
        \item 个人歌单:
            在``个人歌单''栏目中,我们需要展示一个用户自己创建的所有公开歌单。
            所谓公开歌单,就是用户在创建歌单时,可以选择是否他人课件该歌单,
            若用户选择他人可见,则在该栏目中,将显示该歌单。
            其他用户可以对该用户创建的歌单进行点赞操作,并总共的点赞数会显示在界面中。
            具体而言,以下元素需要出现在每一个歌单的信息范围内:
            \begin{itemize}
                \item 歌单名称
                \item 歌单封面
                \item 歌单简介(简短版本)
                \item 点赞按钮,以及总共的点赞次数
            \end{itemize}
        \item 个人动态:
            在``个人动态''栏目中,我们需要展示一个用户发出的所有\emph{动态}。
            所谓动态,即类似于微博或者是微信朋友圈的,集文字、图片、音乐分享为
            一体的体现作者想法的信息。
            同样,这些动态自身可以被其他人看到,并可以进行评论和点赞两种操作。 
            一条动态包含以下元素:
            \begin{itemize}
                \item 文字(当文字过长时,我们在动态视图下仅显示摘要)
                \item 图片(可选)
                \item 音乐(可选,体现为一个缩略图和名称,点击后进入音乐界面)
                \item 音乐集(可选,体现为一个缩略图和名称,点击后进入音乐集界面)
            \end{itemize}        
            注:图片、音乐、音乐集的总数相加限制为9
        \end{enumerate}    
    }
    \R{
    \item \textbf{K歌界面}:
        K歌界面提供了用户K歌相关的内容,一般而言,普通用户使用的K歌录制手段为
        手机麦克风收音,所以我们一般考虑移动版客户端在此功能点的使用情形。
        在界面中,我们需要提供以下界面元素:
        \begin{itemize}
            \item 歌词
            \item 录音控制
            \item 按时间的时间轴控制
            \item 按歌词的时间轴控制
            \item 对录音结果进行编辑、处理的相关按钮
        \end{itemize}
    }
\fi
\end{enumerate}

除此之外,我们还要向第三方开发者提供API,使得第三方开发者可以向他们的网页中
    方便地嵌套音乐播放器,通过引用我们的资源来方便地达到网页音乐播放的功能,
    同时,对于本项目提供推广的同时,也提供了直接在第三方网页上购买歌曲的
    选项。    

\section{用户特征}

在这一小节中,我们讨论\proname 的目标用户具有的画像。
总体而言,我们的产品适合几乎所有用户使用,但是,我们的目标用户有一定的取向,
我们的\textbf{用户画像}大体可划分为以下几类:
\begin{enumerate}
    \item \textbf{学生群体}:时间充裕,活跃度高,追求时尚;
    \item \textbf{年轻白领、IT从业者}:
    时间碎片化,有一定工作压力;
    \item \textbf{音乐从业者}:
    包括艺人、词曲创作人、乐评人、独立音乐人等,其中不乏行业领头人;
    \item \textbf{行业精英/创业者/企业家}:多为高收入人群,事业成熟稳定,年龄偏中年向上;
\end{enumerate}

\section{用户需求} % (fold)

上节我们描述了我们产品的目标用户群体, 
    本节我们分析这些用户的需求,并讨论我们的产品是如何满足这些用户需求的。

根据狩野纪昭发明的针对用户需求分类和排序的KANO模型
, 
    可将用户需求分为五类:基本型需求、期望型需求、魅力型需求、无差异型需求以及反向型需求。
    以下选取前三种需求类型,进行简单举例分析。

\begin{itemize}
    \item 基本型需求:
    \begin{itemize}
        \item 播放音乐
        \item 搜索音乐
        \item 上传音乐
        \item 查看音乐信息
        \item 下载、收藏、购买音乐
        \item 账号管理
    \end{itemize}
    \item 期望型需求:
    \begin{itemize}
        \item 批量管理音乐或歌单
        \item 订阅会员来收听大量付费音乐
        \item 对音乐评分、评论
        \item 查看音乐排行榜
    \end{itemize}
    \item 魅力型需求:
    \begin{itemize}
        \item 个性化智能推荐音乐、音乐集
        \item 与其他用户产生互动
    \end{itemize}
\end{itemize}
\iffalse
\R{ 
随着社交网络进一步渗透到中国以及全世界人们的生活的方方面面,我们
作为一款音乐播放系统,也需要拥抱变化,改善体验,迎接挑战。因此,经过需
求修订,我们加入了围绕着社交网络的一系列项目目标,如:好友系统、私人空
间的分享、评论、点赞系统。}

\R {
通过市场调查,我们可以发现 K 歌功能收到了中国
市场的男女老少的全方面的欢迎,所以,我们也将这些需求放置在我们项目的目
标之中。
}
\fi
\section{假设和依赖关系}

本节我们给出本项目中可能影响SRS中需求的所有的假设因素,如
    准备使用的第三方或商业组件,操作和开发环境的问题约束等。

首先,网页客户端依赖各平台的浏览器,
    我们需要掌握Web开发技术的人员来开发和维护这一部分;
其次,各个平台的软件客户端,需要相关的软件开发人员来进行开发与维护;
而车载娱乐系统以及电视机机顶盒等平台,则需要相关的厂商开放对应的平台给第三方软件,
    否则,我们应先与对应的厂商确定合作关系。

对于音乐的来源,我们需要与各大音乐版权数据库协商取得授权,
    根据各个国家与地区的版权法以及相关法律法规的规定,我们要调整软件在不同位置的工作状态。

对于服务器端,我们需要相关开发人员有丰富的服务器开发能力,并且需要足够的运维人员支撑软件长时间不间断的运行;服务器端还依赖于数据库系统、云服务和相关网络处理的技术,具体的依赖关系可能在上线之后继续更迭。
