%!TEX root = ../main.tex

\begin{abstract}
% 本文是软件工程需求规格说明书模板,修改自于中国科学技术大学本硕博毕业论文 \LaTeX{} 模板示例文件,该模板由
% zepinglee和seisman创建,遵循中国科学技术大学的论文写作规范,适用于撰写学士、硕士和博士学位论文。

本文是2019年春季学期软件工程课程的项目需求分析文档,
	本项目组为课程的第51小组,参与人员为蒋越(组长)、
	吴昊和莫定衡,本项目组选题为《音乐播放系统》,
	经商讨,我们将项目名制定为\emph{\proname}。

% 本文档最后一章演示如何使用 \LaTeX{} 的一些基本命令以及本模板提供的一些特殊功能,
% 模板的选项及详细用法请参考模板说明文档 ustcthesis.pdf。请在提交之前把最后一掌实例注释掉。

本文档说明了\proname 的功能需求、性能需求、外部接口需求
	以及项目总体的设计约束。
同时,我们还针对软件质量特性对本项目的各方面提出了要求,
	并明确了本项目与其他项目或者库的依赖关系。
最后,我们对本文档中所提及的需求说明按照优先级进行分级,
	为项目的实现过程制定了总体规划。
\iffalse
\R{
针对需求进行的更改与新增,我们在该版本中的需求分析中,也进行了
对应的更新。
在项目的实施过程中,可以通过进行这些需求更新的对比来确定新的实施细则。
}
\fi
\keywords{
	软件工程\zhspace{}
	音乐\zhspace{}
	多媒体播放技术\zhspace{}
	网页设计\zhspace{}
	Web\zhspace{}
	云同步\zhspace{}\\
	排名系统\zhspace{}
	社区互动\zhspace{}
	项目需求
}

\begin{table}[htbp]
\centering
\caption{缩略词清单} \label{tab:abbr}
\begin{tabular}{|c|c|c|}
    \hline
    缩略语 & 英文全名 & 中文解释 \\
    \hline
    API & Application Programming Interface & 应用程序接口\\
    \hline
    UI & User Interface & 用户接口\\
    \hline
\end{tabular}
\end{table}

\end{abstract}
