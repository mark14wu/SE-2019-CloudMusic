\chapter{可行性分析结果}

\section{市场分析}

用户音乐口味的多元化。从语种的角度来说,除了中英文歌获得广泛接受以外,日韩及其他音乐也有相当大的一部分人喜欢。从流行与小众的角度来说,小众歌手得到发掘,越来越受到用户的重视。只要是好听的歌都能获得用户的青睐,网络歌手在逐步获得用户的认可。

大部分用户不满足于现有音乐(或多或少的歌荒),开始寻找新的好听的音乐。且出现多种淘歌途径。

准确推荐音乐用户将会非常喜欢,但用户对于目前的推荐功能并不满意。
用户的付费意愿不高,中国互联网用户已习惯免费模式。但有一部分用户开始尝试付费。

用户选择音乐软件时,先入为主的观念较浓,认为音乐软件差不多的不在少数。
乐库里找不到想要歌曲时,用户反映普遍失落。

\section{技术可行性分析}

\proname 使用或依赖的软件和技术,如 HTML 、 Javascript 、PostgreSQL , 均为成熟的技术,有稳定的接口和活跃的社区支持。

\proname 的负责人具有使用这些工具的经验。

\section{知识产权分析}

\proname 使用的软件和技术,均基于开源协议发布,或为公开标准,因此不具有知识产权上的侵犯。

库中的曲目均以合法途径购买并获得使用许可;对用户上传的歌曲,积极进行检查,发现违法的立即下架,则该服务不构成对知识产权的侵犯。

