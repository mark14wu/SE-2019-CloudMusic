\chapter{软件质量特性}

\section {适应性}
\proname 提供了全平台的服务。

\section {可用性}
\proname 提供了傻瓜式的UI界面。

\section {易学性}
\proname 提供了傻瓜式的UI界面。 \proname 的客户端附有对首次使用的用户的指引。

\section {正确性}
\proname 的服务器端应对客户端的请求使用数据包校验和用户信息的比对,拒绝非法的请求。
\proname 的客户端在做出对用户账户的获取和变动之前,应与服务器通信比对用户凭据是否有效。

\section {灵活性}
\proname 的客户端减少各个模块间的依赖,保证逻辑的简洁,并提供简单、全面的接口。使得开发和维护过程中可以根据需求的变更快速更新。

\section {可维护性}
\proname 的代码应有良好的文档和丰富的注释(占总代码的至少 30\%),加上简洁的逻辑,使得整个服务易于维护。

\section {可移植性}
\proname 的网页客户端代码使用标准的 HTML5 和 Javascript  ;使用响应性网页设计使得在任意分辨率和任意支持 HTML5 的浏览器上都能正常显示。使用跨平台的框架(如 Qt ),并在操作系统变动时能尽可能地减小修改。

\section {可测试性}
\proname 的各个模块间应尽可能地减少外部和内部的依赖。如有依赖,应保证尽可能地使用稳定的 API ,并可以使用 mock object 实现单元测试。

\section {健壮性}
\proname 的客户端和服务器端应检测可能发生的错误并处理。服务器端保证前端和数据库的隔离。

\section {可靠性}
\proname 服务器的数据库支持高并发的访问,客户端支持在离线时的正常使用(播放已下载的歌曲),服务器分布式分布,负载均衡,并定期备份。

